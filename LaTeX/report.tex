\documentclass[12pt,letterpaper,final,oneside,openany,onecolumn]{article} 												
\usepackage[lmargin=3.5cm,rmargin=2.5cm,tmargin=3.0cm,bmargin=3.0cm]{geometry}
\usepackage[latin1]{inputenc}												%european
\usepackage[T1]{fontenc}
\usepackage[]{times}
\usepackage{amsmath}																%math package
\usepackage{graphicx}
\usepackage{multicol}
\usepackage{amssymb}
\usepackage{array}
\usepackage[footnotesize]{subfigure}
\usepackage[font=footnotesize]{caption}
\usepackage{makeidx}
\usepackage{color}
\usepackage[hidelinks]{hyperref}
\usepackage{epstopdf}
\usepackage{tabulary}


%\renewcommand{\baselinestretch}{1}
%\renewcommand{\contentsname}{�ndice General}%{Tabla de Contenidos}
%\renewcommand{\listfigurename}{Lista de Figuras}
%\renewcommand{\listtablename}{\'Indice de Tablas}
%\renewcommand{\chaptername}{Cap�tulo}
%\renewcommand{\bibname}{Bibliograf�a}
\renewcommand\floatpagefraction{.9}
\renewcommand\topfraction{.9}
\renewcommand\bottomfraction{.9}
\renewcommand\textfraction{.1}

\setcounter{totalnumber}{50}
\setcounter{topnumber}{50}
\setcounter{bottomnumber}{50}

% Different font in captions
\newcommand{\captionfonts}{\small}

\makeatletter  % Allow the use of @ in command names
\long\def\@makecaption#1#2{%
  \vskip\abovecaptionskip
  \sbox\@tempboxa{{\captionfonts #1: #2}}%
  \ifdim \wd\@tempboxa >\hsize
    {\captionfonts #1: #2\par}
  \else
    \hbox to\hsize{\hfil\box\@tempboxa\hfil}%
  \fi
  \vskip\belowcaptionskip}
\makeatother   % Cancel the effect of \makeatletter



\newcommand\blfootnote[1]{%
  \begingroup
  \renewcommand\thefootnote{}\footnote{#1}%
  \addtocounter{footnote}{-1}%
  \endgroup
}



\pagenumbering{gobble}


%Inicio de documento
\begin{document}
		
		
		
\noindent		
Universidad Central de Venezuela.\\
Facultad de Ciencias.\\
Postgrado en Computaci�n.\\
Desarrollo de Aplicaciones Distribuidas .\\
\hspace*{\fill} Francisco Sans.

\section*{\center{Robot Web Distribuido: Reporte Proyecto 1}}

\section{An�lisis del problema}


\section{Dise�o de la Soluci�n}

La soluci�n propuesta constar� de un conjunto de componentes distribuidos que se encargar�n de realizar el acceso a los enlaces suministrados.
Para ello, se utilizar�n containers de Docker para separar en un mismo entorno f�sico cada uno de los componentes.
Para la comunicaci�n entre el planificador y los descargadores se utiliz� un servidor REST del lado de los descargadores, servicio que el planificador consumir� como cliente.


Para programa el planificador, se utiliz� el lenguaje de programaci�n Python junto con la librer�a Request~\footnote{http://docs.python-requests.org/en/master/} para el consumo de los servicios.
Se recibir� como entrada un archivo con la enlaces semillas para empezar a realizar

JSON


Para los descargadores se utiliz� el lenguaje de programaci�n Ruby, el cual se utiliza junto a la librer�a Sinatra~\footnote{http://sinatrarb.com/} para montar el servidor REST.
Adem�s, el descargador tendr� un servidor Selenium~\footnote{http://www.seleniumhq.org/} stand-alone para Firefox.
El Selenium se encargar� de hacer el renderizado de la p�gina, para as� poder obtener una imagen fidedigna de la misma.
A partir del resultado obtenido por Selenium, el descargador obtendr� todos los enlaces contenidos en las etiquetas \textsl{<a>} 


get /post

Finalmente los almacenes de datos fueron implementados con MongoDB como una base NoSQL, que se utiliza para almacenar el c�digo fuente de la p�gina tal y como es obtenido por Selenium.

El c�digo fuente del proyecto puede encontrarse ubicado en GitHub~\footnote{https://github.com/franjaviersans/DistributedSystemProject}.




\section{Scripts de Ejecuci�n}

Para facilitar la instalaci�n, ejecuci�n y prueba de la soluci�n, se presentan un conjunto de scripts que contienen un conjunto de comandos para realizar estas tareas de manera autom�tica.
Los \textit{scripts} tienen el siguiente uso:

\begin{itemize}
	\item install\_containers.sh: se encarga de la ejecuci�n de dos DOCKERFILEs, correspondientes a: (1) la imagen de Python instalando la librer�a \textsl{request} para realizar consultas REST; y (2) la imagen del selenium stand-alone con Firefox, instalando Ruby con las librer�as \textit{sinatra}, \textsl{selenium-webdriver}, \textsl{json} y \textsl{mongo}. 
Adem�s, este script tambi�n se encarga de descargar la imagen de mongo sin ning�n complemento adicional.
	\item start\_containers.sh: en este script se inician las im�genes de los descargadores, de las bases de datos y de la red de comunicaci�n.
Para las bases de datos se crean directorios para almacenar la informaci�n de las p�ginas.
	\item start\_client.sh: inicia el container del planificador, que se encargar� de realizar la tarea del robot.
	\item kill\_containers.sh: es utilizado para remover de docker los containers utilizados para el planificador, los descargadores y las bases de datos.
Adicionalmente, tambi�n se elimina la red creada para la communicaci�n y los directorios que contienen las bases de datos, de tal manera de limpiar la computadora de la ejecuci�n del programa.
\end{itemize}
 

\end{document}



